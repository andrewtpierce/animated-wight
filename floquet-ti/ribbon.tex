\documentclass{article}

\usepackage{amsmath}
\usepackage{amsfonts}
\usepackage{slashed}
\usepackage{pdftexcmds}
\usepackage{mathrsfs}
\usepackage{braket}
\usepackage{fullpage}
\usepackage{feynmp-auto}

\newcommand{\ies}[1]{ \begin{equation*} \begin{split} #1 \end{split} \end{equation*}}

\newcommand{\hc}{\text{h.c.}}

\title{Graphene ribbon 1-D band structure}
\author{Andrew Pierce}
\date{May 15, 2015}
\begin{document}
\maketitle

\subsection*{Zigzag ribbon}
The tight-banding hamiltonian for graphene is taken as\footnote{``Electronic properties of graphene''} 
\ies{
  H = -t \sum_{\text{n.n.}, \sigma} 
  \left( a^\dagger_i b_j + \hc \right)
  - t' \sum_{\text{n.n.n.}, \sigma} 
  \left( a^\dagger_i a_j + b^\dagger_i b_j + \hc \right)
}
The zigzag ribbon is taken to be infinite in the $x$ direction and finite in the $y$ direction. The sums are now rewritten as a sum over sets of lattice sites that are translationally invariant in the $x$ direction and finite in the $y$ direction. The set of sites is taken to be one of the ``zigzag'' paths starting at the top of the nanoribbon and ending at the bottom. Suppose it goes from top-right to bottom-left. Abandon the $a^\dagger, b^\dagger$ notation and label the sites sequentially: $a^\dagger_{j, 1, \sigma}, \;  a^\dagger_{j,2, \sigma}, \; \ldots$.

First, the nearest-neighbor sum. Consider set $j$. The odd-numbered sites couple to set $j+1$ and the even-numbered sites couple to set $j-1$. Specifically, $a^\dagger_{j, i, \sigma}$ couples to $a_{j+1, i+1, \sigma}$ if $i$ is odd, and to $a_{j-1, i-1, \sigma}$ if $i$ is even. Additionally, each of the sites along within $j$ will couple to its other neighbors in $j$. 

Suppose there are $N$ full honeycomb cells in the $y$ direction. The nearest-neighbor sum may then be rewritten as
\ies{
  -t \sum_{\text{n.n.}, \sigma} 
  \left( a^\dagger_i b_j + \hc \right)
  &=
  -t \sum_{j, \sigma} 
  \left[ H^I_j + 
    \frac{1}{2} \sum_{i=1}^{N+1}  \left( 
      a^\dagger_{j,2i-1, \sigma} a_{j+1, 2i, \sigma}
      + a^\dagger_{j, 2i, \sigma} a_{j-1, 2i-1, \sigma}
    \right)
    +
    \hc
  \right] 
}
where $H^I_j$ is the contribution from tunnelling between lattice sites on the same $j$:
\ies{
  H^I_j = a^\dagger_{j,1,\sigma} a_{j, 2, \sigma} + a^\dagger_{j, 2, \sigma} a_{j, 3, \sigma} + \ldots a^\dagger_{j, 2N+1, \sigma} a_{j, 2 N + 2, \sigma}.
}
In Fourier space, the nearest-neighbor terms takes the form
\ies{
  H_{\text{n.n.}}
  &=
  -t \sum_{k, \sigma}
  \left[
    H^I_k 
    +
    \frac{1}{2}
    \sum_{i=1}^{N+1}
    \left(
      a^\dagger_{k, 2i-1, \sigma} a_{k, 2i, \sigma} e^{ik \sqrt{3}a/2}
      +
      a^\dagger_{k, 2i, \sigma} a_{k, 2i-1, \sigma} e^{-ik \sqrt{3}a/2}
    \right)
    + \hc
  \right]
  \\
  &=
  -t \sum_{k, \sigma} 
  \left[
    (H^I_k + \hc) + \sum_{i=1}^{N+1} \left(
      a^\dagger_{k, 2i-1, \sigma} a_{k, 2i, \sigma} e^{ik \sqrt{3}a/2}
      +
      a^\dagger_{k, 2i, \sigma} a_{k, 2i-1, \sigma} e^{-ik \sqrt{3}a/2}
    \right)
  \right].
}
Here,
\ies{
  H^I_k 
  &=
  a^\dagger_{k,1,\sigma} a_{k, 2, \sigma} 
  +
  a^\dagger_{k, 2, \sigma} a_{k, 3, \sigma} 
  + \ldots +
  a^\dagger_{k, 2N+1, \sigma} a_{k, 2N+2, \sigma}
}
Hence, the $(2N+2)\times(2N+2)$ matrix representation of $H_{\text{n.n.}}(k)$ takes the form
\ies{
  H_{\text{n.n.}}
  &=
  -t \left( \begin{array}{cccccc}
           0 & 1 + e^{i k \sqrt{3} a /2} & 0 & \; & 0 & 0 \\
           1 + e^{-i k \sqrt{3} a /2} & 0 & 1 + e^{i k \sqrt{3} a /2}& \ldots & 0 & 0 \\
           0 & 1 + e^{-i k \sqrt{3} a /2} & 0 & \; & 0 & 0 \\
           \; & \vdots & \; & \ddots & \;  & \; \\
           0 & 0 & 0 & \; & 0 & 1+e^{ik \sqrt{3} a/2} \\
           0 & 0 & 0 & \;  & 1+e^{-ik \sqrt{3} a/2}&0 \\
  \end{array} \right).
}
The eigenvalues of this Toeplitz tridiagonal matrix may be computed analytically. The determinant of the first $3\times3$ block is given by
\ies{
  - \lambda^2 - 2 |a|^2 \lambda,
}
where $a= 1+e^{i k \sqrt{3}a/2}$.
\end{document}