\documentclass{article}

\usepackage{amsmath}
\usepackage{amsfonts}
\usepackage{slashed}
\usepackage{pdftexcmds}
\usepackage{mathrsfs}
\usepackage{braket}
\usepackage{fullpage}
\usepackage{feynmp-auto}

\newcommand{\ies}[1]{ \begin{equation*} \begin{split} #1 \end{split} \end{equation*}}

\title{Floquet TI calculations}
\author{Andrew Pierce}
\date{}
\begin{document}
\maketitle
\section*{PRB 79 081406, 2009}
\subsection*{III. Calculation of Floquet bands for a Dirac band}
For the two dimensional Dirac band, the hamiltonian is\footnote{firt paragraph, section 3}
\ies{
  H(t) = \tau_z v [ k^x + A^x_{ac} (t) ] \sigma_x + v [ k^y + A^y_{ac} (t) ] \sigma_y.
}
$\tau_z = \pm 1$ is a generalized ``valley'' index, v is the velocity, and $\vec{\sigma}$ are the Pauli matrices. The Schrodinger equation reads
\ies{
  H(t) \ket{\Psi_\alpha (t) } = i \frac{\partial}{\partial t} \ket{\Psi_\alpha (t)}.
}
This may be rearranged using the notation of Floquet theory; introduce $\mathcal{H} (t) = H(t) - i \partial_t$ to obtain
\ies{
  \mathcal{H} (t) \ket{\Psi_\alpha t} = 0.
}
Floquet's theorem (equivalent to Bloch's theorem for spatially periodic systems) says that for time periodic $H(t)$ (in this case, time-periodic $A_{ac}$), there exists a set of states $\ket{\Phi_\alpha (t)}$ which are periodic:
\ies{
  \ket{\Phi_\alpha (t + T) } = \ket{\Phi_\alpha (t)},
}
and which satisfy the Floquet equation:
\ies{
  \mathcal{H}(t) \ket{\Phi_\alpha (t)} = \epsilon_\alpha \ket{\Phi_\alpha (t)},
}
where $\epsilon_\alpha$ are known as quasienergies.

Now express the ``Floquet functions''\footnote{by analogy with Bloch functions??} by their Fourier components:
\ies{
  \ket{\Phi_\alpha (t) } = \sum_m e^{- i m \Omega t} \ket{ u_\alpha^m},
}
for some kets $\ket{u}$. 

Re-expanding the Floquet operator $\mathcal{H}$ in terms of $H$ and $\partial_t$ gives
\ies{
  \sum_m H(t) e^{ -i m \Omega t} \ket{ u_\alpha^m } 
  =
  \sum_m \left[ \epsilon_\alpha + m \Omega \right] e^{-i m \Omega t} \ket{ u_\alpha^m }
}
Choose $A$ to be a circularly polarized field: $\vec{A} = A ( \cos \Omega t, \sin \Omega t)$. Here, $A = F/ \Omega$ and $F$ is the ``field strength.''\footnote{?? units} Then the equation reads
\ies{
  \sum_m \left\{
  \tau_z v [ k^x + A \cos \Omega t ] \sigma_x + v [ k^y + A \sin \Omega t (t) ] \sigma_y
    \right\} e^{ -i m \Omega t} \ket{ u_\alpha^m } 
  =
  \sum_m \left[ \epsilon_\alpha + m \Omega \right] e^{-i m \Omega t} \ket{ u_\alpha^m }.
}
Integration against $e^{ i n \Omega t}$ yields
\ies{
  &{}
    \sum_m 
    \frac{1}{T} \int_0^T dt \, 
  \left\{
  \tau_z v [ k^x + A \cos \Omega t ] \sigma_x + v [ k^y + A \sin \Omega t (t) ] \sigma_y
    \right\}  e^{ -i (m-n) \Omega t} \ket{ u_\alpha^m } 
    \\
    \qquad \qquad
  &=
  \sum_m  \frac{1}{T} \int_0^T dt \, \left[ \epsilon_\alpha + m \Omega \right] e^{-i (m-n) \Omega t} \ket{ u_\alpha^m }
}
The needed integrals are 
\ies{
  \frac{1}{T} \int_0^T dt \, \cos \Omega t \,  e^{-i (m - n ) \Omega t}
  &=
  \frac{1}{2T} \int_0^T dt \, \left( e^{ i (n - m + 1 ) \Omega t} + e^{i ( n - m - 1) \Omega t} \right)
  \\
  &=
  \frac{1}{2} \left( \delta_{m, n+1} + \delta_{m, n- 1} \right)
}
\ies{
  \frac{1}{T} \int_0^T dt \, \sin \Omega t \,  e^{-i (m - n ) \Omega t}
  &=
  \frac{1}{2iT} \int_0^T dt \, \left( e^{ i (n - m + 1 ) \Omega t} - e^{i ( n - m - 1) \Omega t} \right)
  \\
  &=
  \frac{1}{2i} \left( \delta_{m, n+1} - \delta_{m, n- 1} \right)
}
The Fourier-transformed Floquet equation becomes
\ies{
  &{} \sum_m \left\{
    \tau_z v \left[ 
      k^x \delta_{m,n} + \frac{1}{2} A (\delta_{m,n+1} + \delta_{m,n-1})
    \right] \sigma_x
    + v \left[
      k^y + \frac{1}{2i} A (\delta_{m,n+1} - \delta_{m,n-1} )
    \right] \sigma_y
  \right\}
  \\
  &\qquad
  = 
  \sum_m
  \left[ \epsilon_\alpha + m\Omega \right] \delta_{m,n} \ket{u^m_\alpha }
}
\ies{
  \Rightarrow
  \left[
    (\tau_z v k^x \sigma_x + v k^y \sigma_y ) \ket{u^n_\alpha}
    +
    \frac{1}{2} Av (\tau_z \sigma_x- i \sigma_y ) \ket{u^{n+1}_\alpha}
    +
    \frac{1}{2} Av (\tau_z \sigma_x + i \sigma_y ) \ket{u^{n-1}_\alpha}
  \right]
  &=
  \left[ \epsilon_\alpha + n \Omega \right] \ket{u^n_\alpha}
}
The left-hand side can be written in the form of a tridiagonal matrix, since the $n$ component is linked to $n-1$ and $n+1$. The $n$th block of this matrix looks like
\ies{
  \left( \begin{array}{ccc}
            v (k^x \tau_z \otimes \sigma_x + k^y 1 \otimes \sigma^y)
           &
           \frac{1}{2} A v (\tau_z \otimes \sigma_x + i 1 \otimes \sigma_y )
           &
           0
           \\
           \frac{1}{2} A v (\tau_z \otimes \sigma_x - i 1 \otimes \sigma_y )
           &
           v (k^x \tau_z \otimes \sigma_x + k^y 1 \otimes \sigma^y)
           &
           \frac{1}{2} A v (\tau_z \otimes \sigma_x + i 1 \otimes \sigma_y ) 
           \\
           0
           &
           \frac{1}{2} A v (\tau_z \otimes \sigma_x - i 1 \otimes \sigma_y )
           &
           v (k^x \tau_z \otimes \sigma_x + k^y 1 \otimes \sigma^y)
  \end{array} \right).
}
% Explicitly,
% \ies{
%   \tau_x \otimes \sigma_x -i 1 \otimes \sigma_y
%   &=
%   \left( \begin{array}{cccc}
%            0&1&0&0\\
%            1&0&0&0\\
%            0&0&0&-1 \\
%            0&0&-1&0 \\
%          \end{array} \right)
%   + \left( \begin{array}{cccc}
%              0&-1&0&0 \\
%              1&0&0&0 \\
%              0&0&0&-1 \\
%              0&0&1&0 \\
%              \end{array} \right).
% }
It is easiest to study the $\tau_z$ bands separately. First takes $\tau_z = +1$. This sector of the matrix takes the form
\ies{
   (\tau_z \otimes \sigma_x  -i 1 \otimes \sigma_y)
  \rightarrow
  \left(
  \begin{array}{cc}
    0&0\\
    2&0
  \end{array}
  \right),
}
\ies{
   (\tau_z \otimes \sigma_x + i 1 \otimes \sigma_y)
  \rightarrow
  \left(
  \begin{array}{cc}
    0&2\\
     0&0
  \end{array}
  \right),
}
\ies{
  (k^x \tau_z \otimes \sigma_x + k^y 1 \otimes \sigma^y)
  \rightarrow
  \left(
  \begin{array}{cc}
    0 & k^x - i k^y \\
    k^x + i k^y & 0 \\
  \end{array}
  \right).
}
On the other hand, the $\tau_z = -1$ sector gives
\ies{
   (\tau_z \otimes \sigma_x  -i 1 \otimes \sigma_y)
  \rightarrow
  \left(
  \begin{array}{cc}
    0&-2\\
    0&0
  \end{array}
  \right),
}
\ies{
   (\tau_z \otimes \sigma_x + i 1 \otimes \sigma_y)
  \rightarrow
  \left(
  \begin{array}{cc}
    0&0\\
     -2&0
  \end{array}
  \right),
}
\ies{
  (k^x \tau_z \otimes \sigma_x + k^y 1 \otimes \sigma^y)
  \rightarrow
  \left(
  \begin{array}{cc}
    0 & k^x - i k^y \\
    k^x + i k^y & 0 \\
  \end{array}
  \right).
}
Setting $v=1$, this block of the matrix equation takes the forms
\ies{
  \tau_z = +1 &: \;
  \left( \begin{array}{cccccc}
           -(n-1)\Omega & k^x - i k^y  & 0 & A & 0  &  0 \\
           k^x + ik^y & -(n-1) \Omega& 0 & 0 & 0 & 0 \\
           0 & 0 &  -n \Omega & k^x - i k^y & 0 & A  \\
           A & 0 & k^x + i k^y & - n \Omega & 0 & 0 \\
           0 & 0 & 0 & 0 & -(n+1) \Omega & k^x - i k^y \\
           0 & 0 & A & 0 & k^x + i k^y & - (n+1) \Omega \\
  \end{array} \right)
  \\
  \tau_z = -1 &: \;
  \left( \begin{array}{cccccc}
           -(n-1)\Omega & k^x - i k^y  & 0 & 0 & 0  &  0 \\
           k^x + ik^y & -(n-1) \Omega& -A & 0 & 0 & 0 \\
           0 & -A &  -n \Omega & k^x - i k^y & 0 & 0  \\
           0 & 0 & k^x + i k^y & - n \Omega & -A & 0 \\
           0 & 0 & 0 & -A & -(n+1) \Omega & k^x - i k^y \\
           0 & 0 & 0 & 0 & k^x + i k^y & - (n+1) \Omega \\
  \end{array} \right)
}
Now the ``Floquet band structure'' may be approximately evaluated numerically by diagonalizing a section of this matrix in the vicinity of $n=0$. 
\end{document}